\chapter{Аналитическая часть}

В данном разделе приведена формализация задачи и данных, рассмотрены типы пользователей и требуемый функционал.  Представлен анализ способов хранения данных и систем управления базами данных, а также произведен выбор оптимальной для решения поставленной задачи системы управления базой данных. 

\section{Формализация задачи}

Необходимо спроектировать и реализовать базу данных для онлайн-мониторинга состояния трасс и подъемников горнолыжного курорта. Также необходимо разработать интерфейс, позволяющий работать с данной базой для получения и изменения хранящейся в ней информации и мониторинга очередей к подъемникам в онлайн-режиме. Требуется реализовать, как минимум, три вида ролей – пользователь, сотрудник лыжного патруля и администратор.

\section{Формализация данных}\label{data}

База данных должна хранить информацию о:
\begin{itemize}
	\item трассах;
	\item подъемниках;
	\item связях трасс и подъемников (на одном подъемнике можно добраться до нескольких трасс, и до одной трассы можно добраться на нескольких подъемниках);
	\item турникетах;
	\item проездных картах;
	\item считываниях карт на турникетах подъемников;
	\item сообщениях о происшествиях;
	\item пользователях;
	\item группах пользователей.
\end{itemize}

В таблице \ref{tbl:1} приведены категории и сведения о данных.

\captionsetup{justification=raggedleft,singlelinecheck=off}
\begin{table}[H]
	\centering
	\caption{Категории и сведения о данных}
	\label{tbl:1}
	\begin{tabular}{|l|l|}
		\hline
		\textbf{Категория}                                                                   & \textbf{Сведения}                                                                                                                                       \\ \hline
		Трассы                                                                               & \begin{tabular}[c]{@{}l@{}}ID трассы, название трассы, уровень\\ сложности, открытость/закрытость.\end{tabular}                                         \\ \hline
		Подъемники                                                                           & \begin{tabular}[c]{@{}l@{}}ID подъемника, название подъемника,\\ открытость/закрытость, количество мест,\\ время подъема, время в очереди.\end{tabular} \\ \hline
		Связи трасс и подъемников                                                            & ID записи, ID подъемника, ID трассы.                                                                                                                    \\ \hline
		Турникеты                                                                            & \begin{tabular}[c]{@{}l@{}}ID турникета, ID подъемника, \\ открытость/закрытость.\end{tabular}                                                          \\ \hline
		Проездные карты                                                                      & ID карты, дата и время активации, тип.                                                                                                                  \\ \hline
		\begin{tabular}[c]{@{}l@{}}Считывания карт на \\ турникетах подъемников\end{tabular} & \begin{tabular}[c]{@{}l@{}}ID записи, ID турникета, ID карты, \\ дата и время считывания.\end{tabular}                                                  \\ \hline
		\begin{tabular}[c]{@{}l@{}}Сообщения о \\ происшествиях\end{tabular}                 & \begin{tabular}[c]{@{}l@{}}ID сообщения, ID отправителя, \\ ID прочитавшего, текст сообщения.\end{tabular}                                              \\ \hline
		Пользователи                                                                         & \begin{tabular}[c]{@{}l@{}}ID пользователя, ID карты, email (логин), \\ пароль, ID группы пользователей.\end{tabular}                                   \\ \hline
		Группы пользователей                                                                 & ID группы пользователей, права доступа.                                                                                                                 \\ \hline
	\end{tabular}
\end{table}

\section{Типы пользователей}\label{functions}

В соответствии с поставленной задачей необходимо разработать приложение с возможностью аутентификации пользователей, что делит их, прежде всего, на авторизованных и неавторизованных. Для управления приложением необходима ролевая модель: авторизованный (обычный) пользователь, сотрудник лыжного патруля и администратор. 

Для каждого типа пользователя предусмотрен свой набор функций:

\begin{itemize}
	
	\item неавторизованный пользователь:
	\begin{itemize}
		\item регистрация,
		\item аутентификация,
		\item просмотр информации о состоянии трасс и подъемников,
		\item просмотр информации о связях трасс и подъемников;
	\end{itemize}
	
	\item авторизованный пользователь:
	\begin{itemize}
		\item выход,
		\item просмотр информации о состоянии трасс и подъемников,
		\item просмотр информации о связях трасс и подъемников,
		\item отправка сообщений о происшествиях;
	\end{itemize}
	
	\item сотрудник лыжного патруля:
	\begin{itemize}
		\item выход,
		\item просмотр и изменение информации о состоянии трасс и подъемников,
		\item просмотр и изменение информации о связях трасс и подъемников,
		\item просмотр сообщений о происшествиях;
	\end{itemize}
	
	\item администратор:
	\begin{itemize}
		\item выход,
		\item просмотр и изменение всей 
		информации, доступной в базе данных, в том числе права доступа групп и отдельных пользователей.
	\end{itemize}
\end{itemize}

\section{Анализ баз данных и систем управления базами данных}

Для реализации поставленной задачи необходимо выбрать подходящую базу данных (БД) и систему управления базой данных (СУБД). 

БД -- это упорядоченный набор структурированной информации или данных, которые обычно хранятся в электронном виде в компьютерной системе. СУБД -- это совокупность программных и лингвистических средств общего или специального назначения, обеспечивающих управление созданием и использованием баз данных \cite{database}.

\subsection{Классификация баз данных по месту хранения информации}

По месту хранения информации БД можно разделить на \cite{inmemory}:
\begin{itemize}
	\item традиционные, которые хранят информацию на жестком диске или другом постоянном носителе; 
	\item in-memory databases (IMDB) (резидентные базы данных), которые хранят информацию непосредственно в оперативной памяти.
\end{itemize}

IMDB появились как ответ традиционным БД в связи со снижением стоимости оперативной памяти, что позволяет хранить весь набор операционных данных непосредственно в памяти, увеличивая тем самым скорость их обработки более чем в 1000 раз \cite{why}.

Ключевыми преимуществами IMDB, в сравнении с традиционными БД, считаются следующие \cite{adv}:

\begin{itemize}
	\item быстрота выполнения операций;
	\item эффективное сохранение зафиксированных данных, которые используются не часто, на жестком диске;
	\item высокая пропускная способность систем, критичных к производительности.
\end{itemize}

Обратной стороной этих достоинств являются следующие недостатки:
\begin{itemize}
	\item однопоточность и эффективная утилизация только одного ядра ЦП, что не позволяет в полной мере воспользоваться возможностями современных многоядерных серверов;
	\item энергозависимость и привязка к размеру оперативной памяти.
\end{itemize}

В практическом плане IMDB-системы особенно востребованы в тех приложениях работы с данными в реальном времени, где требуется минимальное время отклика \cite{lookslike}.

Основным требованием к разрабатываемой БД является предоставление возможности \textbf{онлайн}-мониторинга состояния объектов горнолыжного курорта. То есть 	задача предполагает постоянное добавление и изменение данных, а также быструю отзывчивость на запросы пользователя.

Таким образом, задача является типовым примером использования in-memory БД. И поскольку в современных СУБД существуют надежные и достаточно простые способы устранения указанных недостатков IMDB, было принято решение использовать именно этот подход к хранению данных.



\subsection{Обзор in-memory СУБД}

\noindent\textbf{Memcached}\\

Memcached - это высокопроизводительная система кэширования данных в оперативной памяти, предназначенная для использования в ускорении динамических веб-приложений за счет уменьшения нагрузки на базу данных. Memcached относится к семейству решений для управления данными NoSQL и основана на модели данных с ключевыми значениями  \cite{memcash}. 

Данная СУБД спроектирована так, чтобы все операции имели алгоритмическую сложность O(1), то есть время выполнения любой операции не зависит от количества хранящихся ключей. Это означает, что некоторые операции или возможности, реализация которых требует всего лишь линейного (O(n)) времени, в ней отсутствуют. Так, например, в Memcached отсутствует возможность группировки ключей.


Memcached не является надежным хранилищем – возможна ситуация, когда ключ будет удален из кэша раньше окончания его срока жизни. 

Управление внутренней памятью Memcached более эффективно в простейших случаях использования (при кэшировании относительно небольших и статических данных), поскольку оно потребляет сравнительно мало ресурсов памяти для метаданных. Строки (единственный тип данных, поддерживаемый Memcached) идеально подходят для хранения данных, которые только читаются, потому что строки не требуют дальнейшей обработки. Тем не менее, эффективность управления памятью Memcached быстро уменьшается, когда размер данных является динамическим, после чего память Memcached может стать фрагментированной \cite{memcash2}. 

Еще одно преимущество Memcached -- достаточно простая масштабируемость: поскольку данная система многопоточна, ее можно увеличить, просто предоставив больше вычислительных ресурсов. Однако это может привести к потере части или всех кэшированных данных (в зависимости от того, используется ли постоянное хеширование). \\


\noindent\textbf{Redis}\\


Redis \cite{redis} -- резидентная система управлениями базами данных класса NoSQL с открытым исходным кодом. 

Основной структурой данных, с которой работает Redis является структура типа <<ключ-значение>>, причем значения могут быть пяти различных типов. Данная СУБД используется как для хранения данных, так и для реализации кэшей и брокеров сообщений.

Redis хранит данные в оперативной памяти и снабжена механизмом <<снимков>> и журналирования, что обеспечивает постоянное хранение данных. Существует поддержка репликации данных типа master-slave, транзакций и пакетной обработки комманд.

Redis позволяет осуществлять мелкомасштабный контроль за вытеснением данных, предоставляя выбор из шести различных политик вытеснения \cite{redis2}. 

К недостаткам Redis можно отнести отсутствие вторичных индексов и триггеров. Также транзакции в данной СУБД не удовлетворяют свойствам ACID (Atomicity -- Атомарность, Consistency -- Согласованность, Isolation -- Изолированность, Durability -- Долговечность) \cite{acid}.\\



\noindent\textbf{Tarantool}\\


Tarantool \cite{tarantool} -- это платформа in-memory вычислений с гибкой схемой хранения данных для эффективного создания высоконагруженных приложений. Включает себя базу данных и сервер приложений на языке программирования Lua \cite{lua}.

Записи в Tarantool хранятся в пространствах (space) -- аналог таблицы в реляционной базе данных SQL. Внутри пространства находятся кортежи (tuples), которые похожи на строку в таблице SQL. 


Tarantool объединяет в себе преимущества, характерные для кэша:
\begin{itemize}
	\item <<горячие данные>>;
	\item оптимальная работа при высокой параллельной нагрузке;
	\item низкая задержка (99\% запросов < 1 мс, 99,9\% запросов < 3 мс);
	\item поддерживаемая загрузка на запись — до 1 миллиона транзакций в секунду на одном ядре ЦПУ;
	\item система работает постоянно, не нужно делать перерыв на профилактические работы,
\end{itemize}
и достоинства классических СУБД:
\begin{itemize}
	\item персистентность;
	\item транзакции со свойствами ACID;
	\item наличие репликации (master-slave и master-master);
	\item наличие хранимых процедуры.
	\item поддержка первичных и вторичных индексов (в том числе, составных).
\end{itemize}


В Tarantool реализован механизм <<снимков>> текущего состояния хранилища и журналирования всех операций, что позволяет восстановить состояние базы данных после ее перезагрузки.

К недостаткам данной СУБД можно отнести относительно малое количество поддерживаемых языков (C, C\#, C++, Erlang, Go, Java, JavaScript, Lua, Perl, PHP, Python, Rust) \cite{tarantool}, а также более высокий порог входа по сравнению с ранее рассмотренными СУБД \cite{redis2}.


\subsection{Выбор СУБД для решения задачи}

В данной работе не предполагается хранение в БД очень большого количества информации, и в этом случае требованию о коротком времени отклика удовлетворяют все рассмотренные СУБД.

Для удобного хранения данных, указанных в пункте \ref{data}, СУБД должна предоставлять несколько типов данных, таких как строки, целые числа, числа с плавающей запятой. Этому требованию не удовлетворяет Memcached.
  
Для того, чтобы предоставить пользователям перечисленные в пункте \ref{functions} функции, в СУБД должна быть реализована возможность создания вторичных индексов, а также триггеров или хранимых процедур (для выполнения сложных вычислений). Данному требованию не удовлетворяет Redis.

При этом необходимо надежное хранение данных, без риска их потери даже в случае сбоя в системе, а также реализация ACID транзакций. 

Всем перечисленным требованиям удовлетворяет Tarantool, поэтому именно эта СУБД была выбрана для использования в данной работе.


\section*{Вывод}

В данном разделе была проведена формализация задачи и данных, рассмотрены типы пользователей и требуемый функционал.  В результате анализа способов хранения данных и рассмотрения IMDB-систем, в качестве СУБД для данной работы был выбран Tarantool. 

