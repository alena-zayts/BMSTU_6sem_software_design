\chapter{Аналитическая часть}

В данном разделе описана структура рабочей программы дисциплины. Представлен анализ способов хранения данных и систем управления базами данных, оптимальных для решения поставленной задачи. Описаны проблемы кэшированных данных и представлены методы их решения.

\section{Формализация задачи}

Необходимо спроектировать и реализовать базу данных для онлайн-мониторинга состояния трасс и подъемников горнолыжного курорта. Также необходимо разработать интерфейс, позволяющий работать с данной базой для получения и изменения хранящейся в ней информации и мониторинга очередей к подъемникам в онлайн-режиме. Реализовать, как минимум, три вида ролей – пользователь, сотрудник лыжного патруля и администратор.

\section{Формализация данных}

База данных должна хранить информацию о:
\begin{itemize}
	\item трассах;
	\item подъемниках;
	\item связях трасс и подъемников (на одном подъемнике можно добраться до нескольких трасс, и до одной трассы можно добраться на нескольких подъемниках);
	\item турникетах;
	\item проездных картах;
	\item считываниях карт на турникетах подъемников;
	\item сообщениях о происшествиях;
	\item пользователях;
	\item группах пользователей.
\end{itemize}

В таблице \ref{tbl:1} приведены категории и сведения о данных.

\captionsetup{justification=raggedleft,singlelinecheck=off}
\begin{table}[H]
	\centering
	\caption{Категории и сведения о данных}
	\label{tbl:1}
	\begin{tabular}{|l|l|}
		\hline
		\textbf{Категория}                                                                   & \textbf{Сведения}                                                                                                                                       \\ \hline
		Трассы                                                                               & \begin{tabular}[c]{@{}l@{}}ID трассы, название трассы, уровень\\ сложности, открытость/закрытость.\end{tabular}                                         \\ \hline
		Подъемники                                                                           & \begin{tabular}[c]{@{}l@{}}ID подъемника, название подъемника,\\ открытость/закрытость, количество мест,\\ время подъема, время в очереди.\end{tabular} \\ \hline
		Связи трасс и подъемников                                                            & ID записи, ID подъемника, ID трассы.                                                                                                                    \\ \hline
		Турникеты                                                                            & \begin{tabular}[c]{@{}l@{}}ID турникета, ID подъемника, \\ открытость/закрытость.\end{tabular}                                                          \\ \hline
		Проездные карты                                                                      & ID карты, дата и время активации, тип.                                                                                                                  \\ \hline
		\begin{tabular}[c]{@{}l@{}}Считывания карт на \\ турникетах подъемников\end{tabular} & \begin{tabular}[c]{@{}l@{}}ID записи, ID турникета, ID карты, \\ дата и время считывания.\end{tabular}                                                  \\ \hline
		\begin{tabular}[c]{@{}l@{}}Сообщения о \\ происшествиях\end{tabular}                 & \begin{tabular}[c]{@{}l@{}}ID сообщения, ID отправителя, \\ ID прочитавшего, текст сообщения.\end{tabular}                                              \\ \hline
		Пользователи                                                                         & \begin{tabular}[c]{@{}l@{}}ID пользователя, ID карты, email (логин), \\ пароль, ID группы пользователей.\end{tabular}                                   \\ \hline
		Группы пользователей                                                                 & ID группы пользователей, права доступа.                                                                                                                 \\ \hline
	\end{tabular}
\end{table}

\section{Типы пользователей}

В соответствии с поставленной задачей необходимо разработать приложение с возможностью аутентификации пользователей, что делит их, прежде всего, на авторизованных и неавторизованных. для управления приложением необходима ролевая модель: авторизованный (обычный) пользователь, сотрудник лыжного патруля и администратор. 

Для каждого типа пользователя предусмотрен свой набор функций:

\begin{itemize}
	
	\item неавторизованный пользователь:
	\begin{itemize}
		\item регистрация,
		\item аутентификация,
		\item просмотр информации о состоянии трасс и подъемников,
		\item просмотр информации о связях трасс и подъемников;
	\end{itemize}
	
	\item авторизованный пользователь:
	\begin{itemize}
		\item выход,
		\item просмотр информации о состоянии трасс и подъемников,
		\item просмотр информации о связях трасс и подъемников,
		\item отправка сообщений о происшествиях;
	\end{itemize}
	
	\item сотрудник лыжного патруля:
	\begin{itemize}
		\item выход,
		\item просмотр и изменение информации о состоянии трасс и подъемников,
		\item просмотр и изменение информации о связях трасс и подъемников,
		\item просмотр сообщений о происшествиях;
	\end{itemize}
	
	\item администратор:
	\begin{itemize}
		\item выход,
		\item просмотр и изменение всей 
		информации, доступной в базе данных, в том числе права доступа групп и отдельных пользователей.
	\end{itemize}
\end{itemize}

\section{Анализ баз данных и систем управления базами данных}

Для реализации поставленной задачи необходимо выбрать подходящую базу данных (БД) и систему управления базой данных (СУБД). 

БД -- это упорядоченный набор структурированной информации или данных, которые обычно хранятся в электронном виде в компьютерной системе \cite{database}. СУБД -- это совокупность программных и лингвистических средств общего или специального назначения, обеспечивающих управление созданием и использованием баз данных \cite{database}.

\subsection{Классификация баз данных по месту хранения информации}

По месту хранения информации БД можно разделить на \cite{inmemory}:
\begin{itemize}
	\item традиционные, которые хранят информацию на жестком диске или другом постоянном носителе; 
	\item in-memory databases (IMDB) (резидентные базы данных), которые хранят информацию непосредственно в оперативной памяти.
\end{itemize}

IMDB появились как ответ традиционным БД в связи со снижением стоимости оперативной памяти, что позволяет хранить весь набор операционных данных непосредственно в памяти, увеличивая тем самым скорость их обработки более чем в 1000 раз \cite{why}.

Ключевыми преимуществами IMDB, в сравнении с традиционными БД, считаются следующие \cite{adv}:

\begin{itemize}
	\item быстрота выполнения операций;
	\item эффективное сохранение зафиксированных данных, которые используются не часто, на жестком диске;
	\item высокая пропускная способность систем, критичных к производительности.
\end{itemize}

Обратной стороной этих достоинств являются следующие недостатки:
\begin{itemize}
	\item однопоточность и эффективная утилизация только одного ядра ЦП, что не позволяет в полной мере воспользоваться возможностями современных многоядерных серверов;
	\item энергозависимость и привязка к размеру оперативной памяти.
\end{itemize}

В практическом плане IMDB-системы особенно востребованы в тех приложениях работы с данными в реальном времени, где требуется минимальное время отклика \cite{lookslike}.

Основным требованием к разрабатываемой БД является предоставление возможности \textbf{онлайн}-мониторинга состояния объектов горнолыжного курорта. То есть 	задача предполагает постоянное добавление и изменение данных, а также быструю отзывчивость на запросы пользователя.

Таким образом, задача является типовым примером использования in-memory БД. И поскольку в современных СУБД существуют надежные и достаточно простые способы устранения указанных недостатков IMDB, было принято решение использовать именно этот подход к хранению данных.



\section{Обзор in-memory СУБД}

\subsection{Tarantool}

Tarantool \cite{tarantool} -- это платформа in-memory вычислений с гибкой схемой хранения данных для эффективного создания высоконагруженных приложений. Включает себя базу данных и сервер приложений на языке программирования Lua \cite{lua}.

Записи в Tarantool хранятся в пространствах (space) -- аналог таблицы в реляционной базе данных SQL. Внутри пространства находятся кортежи (tuples), которые похожи на строку в таблице SQL. 


Tarantool объединяет в себе преимущества, характерные для кэша:
\begin{itemize}
	\item <<горячие данные>>;
	\item оптимальная работа при высокой параллельной нагрузке;
	\item низкая задержка (99\% запросов < 1 мс, 99,9\% запросов < 3 мс);
	\item поддерживаемая загрузка на запись — до 1 миллиона транзакций в секунду на одном ядре ЦПУ;
	\item система работает постоянно, не нужно делать перерыв на профилактические работы,
\end{itemize}
и достоинства классических СУБД:
\begin{itemize}
	\item персистентность;
	\item транзакции со свойствами ACID;
	\item наличие репликации (master-slave и master-master);
	\item наличие хранимых процедуры.
	\item поддержка первичных и вторичных индексов (в том числе, составных).
\end{itemize}


В Tarantool реализован механизм <<снимков>> текущего состояния хранилища и журналирования всех операций, что позволяет восстановить состояние базы данных после ее перезагрузки.

\subsection{Redis}

Redis \cite{redis} -- резидентная система управлениями базами данных класса NoSQL с открытым исходным кодом. 

Основной структурой данных, с которой работает Redis является структура типа <<ключ-значение>>. Данная СУБД используется как для хранения данных, так и для реализации кэшей и брокеров сообщений.

Redis хранит данные в оперативной памяти и снабжена механизмом <<снимков>> и журналирования, что обеспечивает постоянное хранение данных. Предоставляются операции для реализации механизма обмена сообщениями в шаблоне <<издатель-подписчик>>: с его помощью приложения могут создавать программные каналы, подписываться на них и помещать в эти каналы сообщения, которые будут получены всеми подписчиками. Существует поддержка репликации данных типа master-slave, транзакций и пакетной обработки комманд.

Все данные Redis хранит в виде словаря, в котором ключи связаны со своими значениями. Ключевое отличие Redis от других хранилищ данных заключается в том, что значения этих ключей не ограничиваются строками. Поддерживаются следующие абстрактные типы данных:

\begin{itemize}
	\item строки;
	\item списки;
	\item множества;
	\item хеш-таблицы;
	\item упорядоченные множества.
\end{itemize}

Тип данных значения определяет, какие операции доступные для него; поддерживаются высокоуровневые операции: например, объединение, разность или сортировка наборов.

\subsection{Выбор СУБД для решения задачи}

Для кэширования данных была выбрана СУБД Tarantool, так как она проста в развертывании и переносимости, и имеет подходящие коннекторы для базы данных PostgreSQL.


\section*{Вывод}

В данном разделе:

\begin{itemize}
 \item рассмотрена структура рабочей программы дисциплины и выявлены её наиболее интересные части;
 \item проанализированы способы хранения информации для система и выбраны оптимальные способы для решения поставленной задачи; 
 \item проведен анализ СУБД, используемых для решения задачи и также выбраны оптимальные информационные системы; 
 \item рассмотрена проблема актуальности кэшируемых данных и предложенно ее решение;
 \item формализованны данные, используемые в системе.
\end{itemize}

