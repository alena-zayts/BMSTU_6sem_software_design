\chapter{Исследовательская часть}

В данном разделе представлена постановка эксперимента по сравнению времени обновления длительности ожидания в очередях к подъемникам в зависимости от их количества и от того, какой объем вычислений переносится на сторону приложения.


\section{Цель эксперимента}

Целью эксперимента является исследование зависимости времени обновления длительности ожидания в очередях к подъемникам от их количества и от того, где производятся основные шаги вычислений: на стороне БД или на стороне приложения.

\section{Технические характеристики}

Технические характеристики устройства, на котором выполнялись измерения:

\begin{itemize}
	\item операционная система: Windows 10;
	\item оперативная память: 16 Гб;
	\item процессор: Intel Core™ i5-8259U.
\end{itemize}

Во время тестирования ноутбук был включен в сеть питания и нагружен только встроенными приложениями окружения и системой тестирования.


\section{Описание эксперимента}

В эксперименте сравнивается время обновления длительности ожидания в очередях к подъемникам в зависимости от их количества и от используемого алгоритма. Для каждого количества подъемников и каждого алгоритма замеры производятся 10 раз, а затем вычисляется среднее время.

\subsection{Сравниваемые алгоритмы}

В эксперименте сравниваются четыре алгоритма обновления времени ожидания в очередях к подъемникам (описаны в порядке убывания объема вычислений, производимых на стороне БД).

В первом варианте алгоритма после получения всех подъемников в цикле для каждого из них вызывается функция update\_queue\_time (см. листинг \ref{lst:func}), которая на уровне БД производит все вычисления: поиск связанных с подъемником турникетов, подсчет считываний на этих турникетах в заданный промежуток времени,  расчет нового времени ожидания и обновление соответствующего кортежа в спейсе lifts). В листинге \ref{lst:upd1} приведен участок кода, реализующий данный алгоритм.

\captionsetup{justification=centering,singlelinecheck=off}
\begin{lstlisting}[label=lst:upd1, caption=Первый вариант алгоритма обновления времени ожидания в очередях к подъемникам, language=csharp]
List<Lift> lifts = _liftsRepository.GetLiftsAsync().GetAwaiter().GetResult(); 
foreach (Lift lift in lifts)
{
	var result = box.Call_1_6<ValueTuple<uint, uint, uint>, Int32[]>("update_queue_time", (ValueTuple.Create(lift.LiftID, (uint)dateFrom.ToUnixTimeSeconds(), (uint)dateTo.ToUnixTimeSeconds()))).GetAwaiter().GetResult();
}
\end{lstlisting}

Во втором варианте после получения всех подъемников в цикле для каждого из них: вызывается функция count\_card\_readings, которая на уровне БД подсчитывает количество считываний карт в заданный промежуток времени на турникетах, связанных с данным подъемником (см. листинг \ref{lst:func}), на уровне приложения осуществляется расчет нового времени ожидания в очереди, происходит вызов функции дя обновления соответствующего кортежа в спейсе lifts. В листинге \ref{lst:upd2} приведен участок кода, реализующий данный алгоритм.

\captionsetup{justification=centering,singlelinecheck=off}
\begin{lstlisting}[label=lst:upd2, caption=Второй вариант алгоритма обновления времени ожидания в очередях к подъемникам, language=csharp]
List<Lift> lifts = _liftsRepository.GetLiftsAsync().GetAwaiter().GetResult();
foreach (Lift lift in lifts)
{
	var result = box.Call_1_6<ValueTuple<uint, uint, uint>, Int32[]>("count_card_readings", (ValueTuple.Create(lift.LiftID, (uint)dateFrom.ToUnixTimeSeconds(), (uint)dateTo.ToUnixTimeSeconds()))).GetAwaiter().GetResult();
	uint cardReadingsAmout = (uint)result.Data[0][0];
	
	uint plusQueueTime = cardReadingsAmout * (2 * lift.LiftingTime / lift.SeatsAmount);
	
	uint newQueueTime = Math.Max(lift.QueueTime - timeDelta + plusQueueTime, 0);
	
	Lift updatedLift = new(lift, newQueueTime);
	_liftsRepository.UpdateLiftByIDAsync(updatedLift.LiftID, updatedLift.LiftName, updatedLift.IsOpen, updatedLift.SeatsAmount, updatedLift.LiftingTime).GetAwaiter().GetResult();
}
\end{lstlisting}


В третьем варианте алгоритма после получения всех подъемников и всех считываний карт происходит отбрасывание считываний, которые не попали в указанный промежуток времени, а затем в цикле для каждого подъемника: путем обращения к БД получаются все турникеты, связанные с данным подъемником, на уровне приложения подсчитывается количество считываний карт в заданный промежуток времени на этих турникетах и осуществляется расчет нового времени ожидания в очереди, а затем происходит вызов функции дя обновления соответствующего кортежа в спейсе lifts. В листинге \ref{lst:upd3} приведен участок кода, реализующий данный алгоритм.

\captionsetup{justification=centering,singlelinecheck=off}
\begin{lstlisting}[label=lst:upd3, caption=Третий вариант алгоритма обновления времени ожидания в очередях к подъемникам, language=csharp]
    List<Lift> lifts = _liftsRepository.GetLiftsAsync().GetAwaiter().GetResult();
List<CardReading> allCardReadings = _cardReadingsRepository.GetCardReadingsAsync()
.GetAwaiter().GetResult();
List<CardReading> cardReadings = new List<CardReading>();
foreach (CardReading cardReading in allCardReadings)
{
	if ((cardReading.ReadingTime >= dateFrom) &&
	(cardReading.ReadingTime < dateTo))
	cardReadings.Add(cardReading);
}

foreach (Lift lift in lifts)
{
	List<Turnstile> turnstiles = _turnstilesRepository.GetTurnstilesByLiftIdAsync(lift.LiftID)
	.GetAwaiter().GetResult();
	List<uint> connectedTurnstileIDs = new List<uint>();
	foreach (Turnstile turnstile in turnstiles)
	{
		connectedTurnstileIDs.Add(turnstile.TurnstileID);
	}
	
	uint cardReadingsAmout = 0;
	foreach (CardReading cardReading in cardReadings)
	{
		if (connectedTurnstileIDs.Contains(cardReading.TurnstileID))
		cardReadingsAmout++;
	}
	
	uint plusQueueTime = cardReadingsAmout * (2 * lift.LiftingTime / lift.SeatsAmount);
	
	uint newQueueTime = Math.Max(lift.QueueTime - timeDelta + plusQueueTime, 0);
	
	Lift updatedLift = new(lift, newQueueTime);
	_liftsRepository.UpdateLiftByIDAsync(updatedLift.LiftID, updatedLift.LiftName, updatedLift.IsOpen, updatedLift.SeatsAmount, updatedLift.LiftingTime).GetAwaiter().GetResult();
}
\end{lstlisting}


В четвертом варианте б\'{о}льшая часть вычислений осуществляется на уровне приложения. Сначала происходит получение всех подъемников, считываний и турникетов из БД. Затем отбрасываются считывания, которые не попали в указанный промежуток времени. Далее в цикле для каждого подъемника: на уровне приложения определяются связанные с ним турникеты, подсчитываются чтения карт на этих турникетах, осуществляется расчет нового времени ожидания в очереди, а затем происходит вызов функции дя обновления соответствующего кортежа в спейсе lifts. В листинге \ref{lst:upd4} приведен участок кода, реализующий данный алгоритм.

\captionsetup{justification=centering,singlelinecheck=off}
\begin{lstlisting}[label=lst:upd4, caption=Четвертый вариант алгоритма обновления времени ожидания в очередях к подъемникам, language=csharp]
    List<Lift> lifts = _liftsRepository.GetLiftsAsync().GetAwaiter().GetResult();
List<Turnstile> turnstiles = _turnstilesRepository.GetTurnstilesAsync()
.GetAwaiter().GetResult();
List<CardReading> allCardReadings = _cardReadingsRepository.GetCardReadingsAsync()
.GetAwaiter().GetResult();
List<CardReading> cardReadings = new List<CardReading>();
foreach (CardReading cardReading in allCardReadings)
{
	if ((cardReading.ReadingTime >= dateFrom) &&
	(cardReading.ReadingTime < dateTo))
	cardReadings.Add(cardReading);
}

foreach (Lift lift in lifts)
{
	List<uint> connectedTurnstileIDs = new List<uint>();
	foreach (Turnstile turnstile in turnstiles)
	{
		if (turnstile.LiftID == lift.LiftID)
		connectedTurnstileIDs.Add(turnstile.TurnstileID);
	}
	
	
	uint cardReadingsAmout = 0;
	foreach (CardReading cardReading in cardReadings)
	{
		if (connectedTurnstileIDs.Contains(cardReading.TurnstileID))
		cardReadingsAmout++;
	}
	
	uint plusQueueTime = cardReadingsAmout * (2 * lift.LiftingTime / lift.SeatsAmount);
	uint newQueueTime = Math.Max(lift.QueueTime - timeDelta + plusQueueTime, 0);
	
	Lift updatedLift = new(lift, newQueueTime);
	_liftsRepository.UpdateLiftByIDAsync(updatedLift.LiftID, updatedLift.LiftName, updatedLift.IsOpen, updatedLift.SeatsAmount, updatedLift.LiftingTime).GetAwaiter().GetResult();
}
\end{lstlisting}

\subsection{Используемые данные}

В поставленном эксперименте для каждого алгоритма измеряется время обновления длительности ожидания в очередях ко всем подъемникам при их различном количестве: 50, 100, 200, 400, 600, 800. Для каждого подъемника при этом создаются 5 турникетов,  для каждого турникета -- по 10 считываний, из которых только половина попадает в заданный промежуток времени. Во всех остальных спейсах хранятся по 1000 записей в каждом.



\section{Результаты эксперимента}

В таблице \ref{tbl:experiment1} представлены результаты поставленного эксперимента. 


\begin{table}[H]
	\centering
	\caption{Результаты сравнения времени (мс), необходимого для обновления длительности ожидания в очередях ко всем подъемникам в зависимости от их количества и используемого алгоритма}
	\label{tbl:experiment1}
	\resizebox{\textwidth}{!}{%
		\begin{tabular}{|l|l|l|l|l|} 
			\hline
			\diagbox{Количество подъемников}{Номер алгоритма} & 1     & 2     & 3     & 4      \\ 
			\hline
			50                                                & 891   & 908   & 1785  & 950    \\ 
			\hline
			100                                               & 1760  & 1797  & 3593  & 1822   \\ 
			\hline
			200                                               & 3535  & 3588  & 7099  & 3600   \\ 
			\hline
			400                                               & 7088  & 7222  & 14186 & 7152   \\ 
			\hline
			600                                               & 10657 & 10754 & 21363 & 10676  \\ 
			\hline
			800                                               & 14175 & 14481 & 28503 & 14306  \\
			\hline
		\end{tabular}%
	}
\end{table}



На рисунке \ref{plt:experiment1} представлены графики зависимости времени обновления длительности ожидания в очередях к подъемникам от их количества и от используемого алгоритма.



\begin{figure}[H]
	\centering
	\begin{tikzpicture}[scale=1.4]
	\begin{axis}[
	axis lines=left,
	xlabel=Количество подъемников,
	ylabel={Время, мс},
	legend pos=north west,
	ymajorgrids=true
	]
	\addplot table[x=N,y=ms,col sep=comma]{csv/exp1.csv};
	\addplot table[x=N,y=ms,col sep=comma]{csv/exp2.csv};
	\addplot table[x=N,y=ms,col sep=comma]{csv/exp3.csv};
	\addplot table[x=N,y=ms,col sep=comma]{csv/exp4.csv};
	\legend{Алгоритм 1, Алгоритм 2, Алгоритм 3, Алгоритм 4}
	\end{axis}
	\end{tikzpicture}
	\captionsetup{justification=centering}
	\caption{Зависимость времени обновления длительности ожидания в очередях к подъемникам от их количества и от используемого алгоритма}
	\label{plt:experiment1}
\end{figure}

По результатам эксперимента видно, что вне зависимости от используемого алгоритма время, затрачиваемое на обновление длительности ожидания в очередях к подъемникам линейно зависит от их количества. 

Времена, затрачиваемые первой, второй и четвертой реализациями, близки между собой, тогда как третья реализация требует сравнительно больше времени.

При этом наименьшее время было затрачено при использовании первого алгоритма, когда все вычисления производятся на стороне БД: в начале к базе данных выполняется одно обращение с получением всех подъемников, а затем для каждого подъемника обращение к ней выполняется только один раз, причем передается небольшой объем информации (3 беззнаковых целых).

Следующая по наименьшему времени обработки -- четвертая реализация. В данном случае между БД и приложением передается больше данных, чем в первой реализации: в приложении дополнительно единожды запрашиваются все считывания и турникеты из БД, а также происходит передача обновленных данных из приложения обратно в БД для каждого подъемника.

Еще больше времени было затрачено второй реализацией, так как там объем передаваемых данных между приложением и БД увеличивается по сравнению с предыдущими: в цикле для каждого подъемника выполняется сразу два обращения к БД. 

Наконец, наибольшее время использовалось в третьей реализации. В этом случае в цикле для каждого подъемника вновь выполняется два обращения к БД. Однако во второй реализации помимо передачи обновленных данных из приложения в БД для каждого подъемника требовалось получить одно число -- количество считываний. В третьей же реализации второе обращение к БД выполняется для получения б\'{о}льшего объема данных: все связанные с подъемником турникеты -- это сразу несколько кортежей.

