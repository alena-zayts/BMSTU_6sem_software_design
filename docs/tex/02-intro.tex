\chapter*{Введение}
\addcontentsline{toc}{chapter}{Введение}

Рабочая программа дисциплины -- программа освоения учебного материала, соответствующая требованиям государственного образовательного стандарта высшего профессионального образования и учитывающая специфику подготовки студентов по избранному направлению или специальности. Разрабатывается для каждой дисциплины учебного плана всех реализуемых в университете основных образовательных программ \cite{rpd-about}.

Хранение, обработка и анализ информации, находящейся в рабочей программы дисциплины может пригодиться для различных систем, например, системы управления обучения (англ. Learning Managment System (LMS) \cite{lms}). Такой интерфейс может предоставить пользователю системы (в данном случае преподавателю) получать и редактировать информацию о рабочей программы дисциплины в режиме онлайн, например, в личном кабинете пользователя.

Рабочая программа дисциплины обычно представлена в виде документа в формате Microsoft Word \cite{ms-word}, что накладывает ограничения на автоматизированную программную обработку и анализ информации, предоставленной в рабочей программы дисциплины.

Цель работы -- реализовать программное обеспечение для хранения, редактирования и удаления данных о рабочих программах дисциплин. 

Чтобы достигнуть поставленной цели, требуется решить следующие задачи:

\begin{itemize}
    \item проанализировать варианты представления данных и выбрать подходящий вариант для решения задачи;
    \item проанализировать системы управления базами данных и выбрать подходящую систему для хранения данных;
    \item спроектировать базу данных, описать ее сущности и связи;
    \item реализовать интерфейс для доступа к базе данных;
    \item реализовать программное обеспечение, которое позволит получить доступ к данным по средствам REST API \cite{rest-api}.
\end{itemize}
