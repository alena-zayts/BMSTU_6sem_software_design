\chapter*{ВВЕДЕНИЕ}
\addcontentsline{toc}{chapter}{ВВЕДЕНИЕ}


В современном мире при решении многих прикладных задач используют различные базы данных. Иногда в больших проектах может использоваться сразу несколько различных баз данных, которые имеют отличную друг от друга архитектуру. Причин для этого может быть много, например, когда два отдельных друг от друга проекта сливаются в один, или, когда для решения различных задач требуются различные по архитектуре базы данных.

В больших проектах различные базы данных обрабатывают информацию, которую иногда следует обрабатывать совместно друг с другом. Например, в одной базе данных хранится информация о готовности товара, а в другой -- о выпуске товара и в обеих базах данных обрабатывается информация о каких-то одинаковых товарах. На сегодняшний день существует очень мало приложений, которые могут работать с данными из различных баз данных, связанно это может быть с уникальностью архитектур различных СУБД \cite{vvedenie}. 

Реляционная модель данных нужна для организации данных в виде таблиц (отношений) с помощью связей - полей таблиц, 
ссылающихся на другие таблицы, которые называются внешними ключами.  
Сегодня существуют другие модели данных, включая NoSQL, но системы управления реляционными базами данных (СУБД) остаются доминирующими для хранения 
и управления данными во всем мире \cite{db-engines-rating}.

Одними из самых популярных СУБД являются MySQL и PostgreSQL \cite{popular}. 

MySQL --- свободная реляционная система управления базами данных. 

PostgreSQL - свободная объектно-реляционная система управления базами данных.
Postgres более функциональная СУБД, нежели MySQL хотя и сложнее для использования.

PostgreSQL является популярным выбором для функций NoSQL. Она изначально    поддерживает   большое   разнообразие типов данных, например таких как JSON, hstore и XML \cite{postgres-types}.

Целью данной курсовой работы является разработка приложения для выполнения распределенного запроса, который использует данные из \newline PostgreSQL и MySQL СУБД.

Для достижения данной цели требуется решить следующие задачи:
\begin{itemize}
	\item проанализировать язык SQL, а также группу операторов DML;
	\item проанализировать совместимость и различия PostgreSQL и MySQL;
	\item проанализировать работу оператора SELECT и табличные выражения;
	\item проанализировать и сформулировать ограничения на операторы JOIN и WHERE; 
	\item разработать программное обеспечение, включающие в себя лексический, семантический и синтаксический анализаторы;
	\item сравнить время выполнения запроса для баз данных, находящихся в различных СУБД и в одной СУБД.
\end{itemize}